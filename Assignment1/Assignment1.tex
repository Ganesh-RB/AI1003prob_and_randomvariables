\documentclass[journal,12pt,twocolumn]{IEEEtran}

\usepackage{setspace}
\usepackage{gensymb}
\singlespacing
\usepackage[cmex10]{amsmath}

\usepackage{amsthm}

\usepackage{mathrsfs}
\usepackage{txfonts}
\usepackage{stfloats}
\usepackage{bm}
\usepackage{cite}
\usepackage{cases}
\usepackage{subfig}

\usepackage{longtable}
\usepackage{multirow}

\usepackage{enumitem}
\usepackage{mathtools}
\usepackage{steinmetz}
\usepackage{tikz}
\usepackage{circuitikz}
\usepackage{verbatim}
\usepackage{tfrupee}
\usepackage[breaklinks=true]{hyperref}
\usepackage{graphicx}
\usepackage{tkz-euclide}

\usetikzlibrary{calc,math}
\usepackage{listings}
    \usepackage{color}                                            %%
    \usepackage{array}                                            %%
    \usepackage{longtable}                                        %%
    \usepackage{calc}                                             %%
    \usepackage{multirow}                                         %%
    \usepackage{hhline}                                           %%
    \usepackage{ifthen}                                           %%
    \usepackage{lscape}     
\usepackage{multicol}
\usepackage{chngcntr}

\DeclareMathOperator*{\Res}{Res}

\renewcommand\thesection{\arabic{section}}
\renewcommand\thesubsection{\thesection.\arabic{subsection}}
\renewcommand\thesubsubsection{\thesubsection.\arabic{subsubsection}}

\renewcommand\thesectiondis{\arabic{section}}
\renewcommand\thesubsectiondis{\thesectiondis.\arabic{subsection}}
\renewcommand\thesubsubsectiondis{\thesubsectiondis.\arabic{subsubsection}}


\hyphenation{op-tical net-works semi-conduc-tor}
\def\inputGnumericTable{}                                 %%

\lstset{
%language=C,
frame=single, 
breaklines=true,
columns=fullflexible
}
\begin{document}


\newtheorem{theorem}{Theorem}[section]
\newtheorem{problem}{Problem}
\newtheorem{proposition}{Proposition}[section]
\newtheorem{lemma}{Lemma}[section]
\newtheorem{corollary}[theorem]{Corollary}
\newtheorem{example}{Example}[section]
\newtheorem{definition}[problem]{Definition}

\newcommand{\BEQA}{\begin{eqnarray}}
\newcommand{\EEQA}{\end{eqnarray}}
\newcommand{\define}{\stackrel{\triangle}{=}}
\bibliographystyle{IEEEtran}
\raggedbottom
\setlength{\parindent}{0pt}
\providecommand{\mbf}{\mathbf}
\providecommand{\pr}[1]{\ensuremath{\Pr\left(#1\right)}}
\providecommand{\qfunc}[1]{\ensuremath{Q\left(#1\right)}}
\providecommand{\sbrak}[1]{\ensuremath{{}\left[#1\right]}}
\providecommand{\lsbrak}[1]{\ensuremath{{}\left[#1\right.}}
\providecommand{\rsbrak}[1]{\ensuremath{{}\left.#1\right]}}
\providecommand{\brak}[1]{\ensuremath{\left(#1\right)}}
\providecommand{\lbrak}[1]{\ensuremath{\left(#1\right.}}
\providecommand{\rbrak}[1]{\ensuremath{\left.#1\right)}}
\providecommand{\cbrak}[1]{\ensuremath{\left\{#1\right\}}}
\providecommand{\lcbrak}[1]{\ensuremath{\left\{#1\right.}}
\providecommand{\rcbrak}[1]{\ensuremath{\left.#1\right\}}}
\theoremstyle{remark}
\newtheorem{rem}{Remark}
\newcommand{\sgn}{\mathop{\mathrm{sgn}}}
%\providecommand{\abs}[1]{\left\vert#1\right\vert}
\providecommand{\res}[1]{\Res\displaylimits_{#1}} 
%\providecommand{\norm}[1]{\left\lVert#1\right\rVert}
%\providecommand{\norm}[1]{\lVert#1\rVert}
\providecommand{\mtx}[1]{\mathbf{#1}}
%\providecommand{\mean}[1]{E\left[ #1 \right]}
\providecommand{\fourier}{\overset{\mathcal{F}}{ \rightleftharpoons}}
%\providecommand{\hilbert}{\overset{\mathcal{H}}{ \rightleftharpoons}}
\providecommand{\system}{\overset{\mathcal{H}}{ \longleftrightarrow}}
	%\newcommand{\solution}[2]{\textbf{Solution:}{#1}}
\newcommand{\solution}{\noindent \textbf{Solution: }}
\newcommand{\cosec}{\,\text{cosec}\,}
\providecommand{\dec}[2]{\ensuremath{\overset{#1}{\underset{#2}{\gtrless}}}}
\newcommand{\myvec}[1]{\ensuremath{\begin{pmatrix}#1\end{pmatrix}}}
\newcommand{\mydet}[1]{\ensuremath{\begin{vmatrix}#1\end{vmatrix}}}
\numberwithin{equation}{subsection}
\makeatletter
\@addtoreset{figure}{problem}
\makeatother
\let\StandardTheFigure\thefigure
\let\vec\mathbf
\renewcommand{\thefigure}{\theproblem}
\def\putbox#1#2#3{\makebox[0in][l]{\makebox[#1][l]{}\raisebox{\baselineskip}[0in][0in]{\raisebox{#2}[0in][0in]{#3}}}}
     \def\rightbox#1{\makebox[0in][r]{#1}}
     \def\centbox#1{\makebox[0in]{#1}}
     \def\topbox#1{\raisebox{-\baselineskip}[0in][0in]{#1}}
     \def\midbox#1{\raisebox{-0.5\baselineskip}[0in][0in]{#1}}
\vspace{3cm}
\title{ASSIGNMENT 1}
\author{Ganesh Bombatkar - CS20BTECH11016}
\maketitle
\newpage
\bigskip
\renewcommand{\thefigure}{\theenumi}
\renewcommand{\thetable}{\theenumi}
Download all python codes from 
\begin{lstlisting}
https://github.com/Ganesh-RB/AI1003prob_and_randomvariables/Assignment1/codes
\end{lstlisting}
%
and latex-tikz codes from 
%
\begin{lstlisting}
https://github.com/Ganesh-RB/AI1003prob_and_randomvariables/Assignment1
\end{lstlisting}
\section{Problem}
Suppose we have four boxes A,B,C and D containing coloured marbles as given below:
\begin{center}
\begin{tabular}{||c c c c||}
    \hline
    Box &Red &White &Black \\
    \hline
    A &1 &6 &3\\
    \hline
    B &6 &2 &2\\
    \hline
    C &8 &1 &1\\
    \hline
    D &0 &6 &4\\
    \hline
\end{tabular}
\end{center}
One of the box has been selected at random and a single marble has been drawn from it. If the marble is red, what is the probability that it was drawn from box A?,box B?, box C?

\section{Solution}
Let $X\in$ \{A,B,C,D\} represents the box and $Y \in$ \{0,1,2\} represents marbles, 0 representing Red, 1 representing White. Then,

\begin{align}
    \pr{X=A}=& \pr{X=B}= \pr{X=C}=\pr{X=D}= \frac{1}{4}
\end{align}
\begin{align}
    \pr{Y=0|X=A}=&\frac{1}{10}
    \\  \pr{Y=0|X=B}=&\frac{3}{5}
    \\  \pr{Y=0|X=C}=&\frac{4}{5}
    \\  \pr{Y=0|X=D}=& 0
\end{align}
Now
\begin{multline}
    \pr{Y=0}= \sum_{k}{\pr{Y=0,X=k}}
    \\ =\sum_{k}{\pr{Y=0|X=k} \times \pr{X=k}}
    = \frac{3}{8}
\end{multline}

Then,
\\For probability of drawn red marble was from box A


\begin{multline}
    \pr{X=A|Y=0}=\frac{\pr{X=A,Y=0}}{\pr{Y=0}}
    \\ =\frac{\pr{Y=0|X=A}\times \pr{X=A}}{\pr{Y=0}}
    \\ =\frac{\frac{1}{10} \times \frac{1}{4}}{\frac{3}{8}}
    =\frac{1}{15}=0.06667
\end{multline}

Similarly , For box B
\begin{multline}
    \pr{X=B|Y=0}=\frac{\pr{Y=0|X=B}\times \pr{X=B}}{\pr{Y=0}}
    \\ =\frac{\frac{3}{5} \times \frac{1}{4}}{\frac{3}{8}}
    =\frac{2}{5} 
    =0.4
\end{multline}

Similarly , for box D

\begin{multline}
    \pr{X=C|Y=0}=\frac{\pr{Y=0|X=C}\times \pr{X=C}}{\pr{Y=0}}
    \\ =\frac{\frac{4}{5} \times \frac{1}{4}}{\frac{3}{8}}
    =\frac{8}{15} 
    =0.53333
\end{multline}

\end{document}