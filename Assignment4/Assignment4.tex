\documentclass[journal,12pt,twocolumn]{IEEEtran}

\usepackage{setspace}
\usepackage{gensymb}
\singlespacing
\usepackage[cmex10]{amsmath}

\usepackage{amsthm}

\usepackage{mathrsfs}
\usepackage{txfonts}
\usepackage{stfloats}
\usepackage{bm}
\usepackage{cite}
\usepackage{cases}
\usepackage{subfig}

\usepackage{longtable}
\usepackage{multirow}

\usepackage{enumitem}
\usepackage{mathtools}
\usepackage{steinmetz}
\usepackage{tikz}
\usepackage{circuitikz}
\usepackage{verbatim}
\usepackage{tfrupee}
\usepackage[breaklinks=true]{hyperref}
\usepackage{graphicx}
\usepackage{tkz-euclide}

\usetikzlibrary{calc,math}
\usepackage{listings}
    \usepackage{color}                                            %%
    \usepackage{array}                                            %%
    \usepackage{longtable}                                        %%
    \usepackage{calc}                                             %%
    \usepackage{multirow}                                         %%
    \usepackage{hhline}                                           %%
    \usepackage{ifthen}                                           %%
    \usepackage{lscape}     
\usepackage{multicol}
\usepackage{chngcntr}

\DeclareMathOperator*{\Res}{Res}

\renewcommand\thesection{\arabic{section}}
\renewcommand\thesubsection{\thesection.\arabic{subsection}}
\renewcommand\thesubsubsection{\thesubsection.\arabic{subsubsection}}

\renewcommand\thesectiondis{\arabic{section}}
\renewcommand\thesubsectiondis{\thesectiondis.\arabic{subsection}}
\renewcommand\thesubsubsectiondis{\thesubsectiondis.\arabic{subsubsection}}


\hyphenation{op-tical net-works semi-conduc-tor}
\def\inputGnumericTable{}                                 %%

\lstset{
%language=C,
frame=single, 
breaklines=true,
columns=fullflexible
}
\begin{document}


\newtheorem{theorem}{Theorem}[section]
\newtheorem{problem}{Problem}
\newtheorem{proposition}{Proposition}[section]
\newtheorem{lemma}{Lemma}[section]
\newtheorem{corollary}[theorem]{Corollary}
\newtheorem{example}{Example}[section]
\newtheorem{definition}[problem]{Definition}

\newcommand{\BEQA}{\begin{eqnarray}}
\newcommand{\EEQA}{\end{eqnarray}}
\newcommand{\define}{\stackrel{\triangle}{=}}
\bibliographystyle{IEEEtran}
\raggedbottom
\setlength{\parindent}{0pt}
\providecommand{\mbf}{\mathbf}
\providecommand{\pr}[1]{\ensuremath{\Pr\left(#1\right)}}
\providecommand{\qfunc}[1]{\ensuremath{Q\left(#1\right)}}
\providecommand{\sbrak}[1]{\ensuremath{{}\left[#1\right]}}
\providecommand{\lsbrak}[1]{\ensuremath{{}\left[#1\right.}}
\providecommand{\rsbrak}[1]{\ensuremath{{}\left.#1\right]}}
\providecommand{\brak}[1]{\ensuremath{\left(#1\right)}}
\providecommand{\lbrak}[1]{\ensuremath{\left(#1\right.}}
\providecommand{\rbrak}[1]{\ensuremath{\left.#1\right)}}
\providecommand{\cbrak}[1]{\ensuremath{\left\{#1\right\}}}
\providecommand{\lcbrak}[1]{\ensuremath{\left\{#1\right.}}
\providecommand{\rcbrak}[1]{\ensuremath{\left.#1\right\}}}
\theoremstyle{remark}
\newtheorem{rem}{Remark}
\newcommand{\sgn}{\mathop{\mathrm{sgn}}}
\providecommand{\abs}[1]{\ensuremath{\left \vert #1\right\vert}}
\providecommand{\res}[1]{\Res\displaylimits_{#1}} 
%\providecommand{\norm}[1]{\left\lVert#1\right\rVert}
%\providecommand{\norm}[1]{\lVert#1\rVert}
\providecommand{\mtx}[1]{\mathbf{#1}}
%\providecommand{\mean}[1]{E\left[ #1 \right]}
\providecommand{\fourier}{\overset{\mathcal{F}}{ \rightleftharpoons}}
%\providecommand{\hilbert}{\overset{\mathcal{H}}{ \rightleftharpoons}}
\providecommand{\system}{\overset{\mathcal{H}}{ \longleftrightarrow}}
	%\newcommand{\solution}[2]{\textbf{Solution:}{#1}}
\newcommand{\solution}{\noindent \textbf{Solution: }}
\newcommand{\cosec}{\,\text{cosec}\,}
\providecommand{\dec}[2]{\ensuremath{\overset{#1}{\underset{#2}{\gtrless}}}}
\newcommand{\myvec}[1]{\ensuremath{\begin{pmatrix}#1\end{pmatrix}}}
\newcommand{\mydet}[1]{\ensuremath{\begin{vmatrix}#1\end{vmatrix}}}
\numberwithin{equation}{subsection}
\makeatletter
\@addtoreset{figure}{problem}
\makeatother
\let\StandardTheFigure\thefigure
\let\vec\mathbf
\renewcommand{\thefigure}{\theproblem}
\def\putbox#1#2#3{\makebox[0in][l]{\makebox[#1][l]{}\raisebox{\baselineskip}[0in][0in]{\raisebox{#2}[0in][0in]{#3}}}}
     \def\rightbox#1{\makebox[0in][r]{#1}}
     \def\centbox#1{\makebox[0in]{#1}}
     \def\topbox#1{\raisebox{-\baselineskip}[0in][0in]{#1}}
     \def\midbox#1{\raisebox{-0.5\baselineskip}[0in][0in]{#1}}
\vspace{3cm}
\title{Assignment 4}
\author{Ganesh Bombatkar - CS20BTECH11016}
\maketitle
\newpage
\bigskip
\renewcommand{\thefigure}{\theenumi}
\renewcommand{\thetable}{\theenumi}
Download all python codes from 
\begin{lstlisting}
https://github.com/Ganesh-RB/AI1103prob-and-randomvariables/Assignment4/codes
\end{lstlisting}
%
and latex-tikz codes from 
%
\begin{lstlisting}
https://github.com/Ganesh-RB/AI1103prob-and-randomvariables/Assignment4
\end{lstlisting}

\section{Problem}
CSIR UGC NET EXAM (Dec 2012) Q 51\\ Suppose $X1,X2,X3,X4$ are i.i.d random variables taking values $1$ and $-1$ with probability $1/2$ each. Then $E\brak{X_1+X_2 +X_3 +X_4}^4$ equals
\begin{multicols}{4}
\begin{enumerate}
    \item $4$
    \item $76$
    \item $16$
    \item $12$
\end{enumerate}
\end{multicols}

\section{Solution}
By defining ${Z}$ transform of $X_i$ as $X_i\sbrak{z}$ for $i \in \cbrak{1,2,3,4} $
\begin{align}
    X_i\sbrak{z}=\frac{1}{2}\cdot z + \frac{1}{2} \cdot z^{-1}=\frac{z+z^{-1}}{2}
\end{align}

Let $Y=X_1+X_2+X_3+X_4$ , then
\begin{align}
    Y\sbrak{z}=& X_1\sbrak{z} \cdot X_2\sbrak{z} \cdot X_3\sbrak{z} \cdot X_4\sbrak{z} 
    \\ =& \brak{\frac{z+z^{-1}}{2}}^4
    \label{eq-z_Transf_Y}
\end{align}

Taking derivatives of $Y\sbrak{z}$
\begin{align}
    \sbrak{\frac{d}{dz}Y\sbrak{Z}}_{z=1} =& E\sbrak{Y}
    \label{eq:d1}
    \\=&\frac{z^{3}}{4} + \frac{z}{2} - \frac{1}{2 z^{3}} - \frac{1}{4 z^{5}}
    \\=&0
\end{align}    
    
\begin{align}
    \sbrak{\frac{d^2}{dz^2}Y\sbrak{Z}}_{z=1} =& E\sbrak{Y^2}-E\sbrak{Y}
    \label{eq:d2}
    \\=&\frac{3 z^{2}}{4} + \frac{1}{2} + \frac{3}{2 z^{4}} + \frac{5}{4 z^{6}}
    \\=&4
\end{align}    
    
\begin{align}
     \sbrak{\frac{d^3}{dz^3}Y\sbrak{Z}}_{z=1} =& E\sbrak{Y^3}-3E\sbrak{Y^2}+2E\sbrak{Y}
     \label{eq:d3}
     \\=&\frac{3 z}{2} - \frac{6}{z^{5}} - \frac{15}{2 z^{7}}
     \\=&-12
\end{align}     
     
\begin{align}
    \begin{split}
         &\sbrak{\frac{d^4}{dz^4}Y\sbrak{Z}}_{z=1} \\&\quad= E\sbrak{Y^4}-6E\sbrak{Y^3}+11E\sbrak{Y^2}-6E\sbrak{Y}
    \end{split}
    \label{eq:d4}
    \\&\quad=\frac{3}{2} + \frac{30}{z^{6}} + \frac{105}{2 z^{8}}
    \\&\quad=84
\end{align}

By manipulating equation \eqref{eq:d1} ,\eqref{eq:d2} ,\eqref{eq:d3} and \eqref{eq:d4}
\begin{multline}
    E\sbrak{Y^4}=\sbrak{\frac{d^4}{dz^4}Y\sbrak{Z}}_{z=1} + 6 \cdot \sbrak{\frac{d^3}{dz^3}Y\sbrak{Z}}_{z=1} \\+ 7 \cdot \sbrak{\frac{d^2}{dz^2}Y\sbrak{Z}}_{z=1} +25 \cdot \sbrak{\frac{d}{dz}Y\sbrak{Z}}_{z=1}
\end{multline}

\begin{align*}
    \mbf{\therefore E\sbrak{Y^4}=84+6\cdot\brak{-12}+7\cdot4+25\cdot0=40}
\end{align*}

\end{document}
