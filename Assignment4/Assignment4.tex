\documentclass[journal,12pt,twocolumn]{IEEEtran}

\usepackage{setspace}
\usepackage{gensymb}
\singlespacing
\usepackage[cmex10]{amsmath}

\usepackage{amsthm}

\usepackage{mathrsfs}
\usepackage{txfonts}
\usepackage{stfloats}
\usepackage{bm}
\usepackage{cite}
\usepackage{cases}
\usepackage{subfig}

\usepackage{longtable}
\usepackage{multirow}

\usepackage{enumitem}
\usepackage{mathtools}
\usepackage{steinmetz}
\usepackage{tikz}
\usepackage{circuitikz}
\usepackage{verbatim}
\usepackage{tfrupee}
\usepackage[breaklinks=true]{hyperref}
\usepackage{graphicx}
\usepackage{tkz-euclide}

\usetikzlibrary{calc,math}
\usepackage{listings}
    \usepackage{color}                                            %%
    \usepackage{array}                                            %%
    \usepackage{longtable}                                        %%
    \usepackage{calc}                                             %%
    \usepackage{multirow}                                         %%
    \usepackage{hhline}                                           %%
    \usepackage{ifthen}                                           %%
    \usepackage{lscape}     
\usepackage{multicol}
\usepackage{chngcntr}

\DeclareMathOperator*{\Res}{Res}

\renewcommand\thesection{\arabic{section}}
\renewcommand\thesubsection{\thesection.\arabic{subsection}}
\renewcommand\thesubsubsection{\thesubsection.\arabic{subsubsection}}

\renewcommand\thesectiondis{\arabic{section}}
\renewcommand\thesubsectiondis{\thesectiondis.\arabic{subsection}}
\renewcommand\thesubsubsectiondis{\thesubsectiondis.\arabic{subsubsection}}


\hyphenation{op-tical net-works semi-conduc-tor}
\def\inputGnumericTable{}                                 %%

\lstset{
%language=C,
frame=single, 
breaklines=true,
columns=fullflexible
}
\begin{document}


\newtheorem{theorem}{Theorem}[section]
\newtheorem{problem}{Problem}
\newtheorem{proposition}{Proposition}[section]
\newtheorem{lemma}{Lemma}[section]
\newtheorem{corollary}[theorem]{Corollary}
\newtheorem{example}{Example}[section]
\newtheorem{definition}[problem]{Definition}

\newcommand{\BEQA}{\begin{eqnarray}}
\newcommand{\EEQA}{\end{eqnarray}}
\newcommand{\define}{\stackrel{\triangle}{=}}
\bibliographystyle{IEEEtran}
\raggedbottom
\setlength{\parindent}{0pt}
\providecommand{\mbf}{\mathbf}
\providecommand{\pr}[1]{\ensuremath{\Pr\left(#1\right)}}
\providecommand{\qfunc}[1]{\ensuremath{Q\left(#1\right)}}
\providecommand{\sbrak}[1]{\ensuremath{{}\left[#1\right]}}
\providecommand{\lsbrak}[1]{\ensuremath{{}\left[#1\right.}}
\providecommand{\rsbrak}[1]{\ensuremath{{}\left.#1\right]}}
\providecommand{\brak}[1]{\ensuremath{\left(#1\right)}}
\providecommand{\lbrak}[1]{\ensuremath{\left(#1\right.}}
\providecommand{\rbrak}[1]{\ensuremath{\left.#1\right)}}
\providecommand{\cbrak}[1]{\ensuremath{\left\{#1\right\}}}
\providecommand{\lcbrak}[1]{\ensuremath{\left\{#1\right.}}
\providecommand{\rcbrak}[1]{\ensuremath{\left.#1\right\}}}
\theoremstyle{remark}
\newtheorem{rem}{Remark}
\newcommand{\sgn}{\mathop{\mathrm{sgn}}}
\providecommand{\abs}[1]{\ensuremath{\left \vert #1\right\vert}}
\providecommand{\res}[1]{\Res\displaylimits_{#1}} 
%\providecommand{\norm}[1]{\left\lVert#1\right\rVert}
%\providecommand{\norm}[1]{\lVert#1\rVert}
\providecommand{\mtx}[1]{\mathbf{#1}}
%\providecommand{\mean}[1]{E\left[ #1 \right]}
\providecommand{\fourier}{\overset{\mathcal{F}}{ \rightleftharpoons}}
%\providecommand{\hilbert}{\overset{\mathcal{H}}{ \rightleftharpoons}}
\providecommand{\system}{\overset{\mathcal{H}}{ \longleftrightarrow}}
	%\newcommand{\solution}[2]{\textbf{Solution:}{#1}}
\newcommand{\solution}{\noindent \textbf{Solution: }}
\newcommand{\cosec}{\,\text{cosec}\,}
\providecommand{\dec}[2]{\ensuremath{\overset{#1}{\underset{#2}{\gtrless}}}}
\newcommand{\myvec}[1]{\ensuremath{\begin{pmatrix}#1\end{pmatrix}}}
\newcommand{\mydet}[1]{\ensuremath{\begin{vmatrix}#1\end{vmatrix}}}
\numberwithin{equation}{subsection}
\makeatletter
\@addtoreset{figure}{problem}
\makeatother
\let\StandardTheFigure\thefigure
\let\vec\mathbf
\renewcommand{\thefigure}{\theproblem}
\def\putbox#1#2#3{\makebox[0in][l]{\makebox[#1][l]{}\raisebox{\baselineskip}[0in][0in]{\raisebox{#2}[0in][0in]{#3}}}}
     \def\rightbox#1{\makebox[0in][r]{#1}}
     \def\centbox#1{\makebox[0in]{#1}}
     \def\topbox#1{\raisebox{-\baselineskip}[0in][0in]{#1}}
     \def\midbox#1{\raisebox{-0.5\baselineskip}[0in][0in]{#1}}
\vspace{3cm}
\title{Assignment 4}
\author{Ganesh Bombatkar - CS20BTECH11016}
\maketitle
\newpage
\bigskip
\renewcommand{\thefigure}{\theenumi}
\renewcommand{\thetable}{\theenumi}
Download all python codes from 
\begin{lstlisting}
https://github.com/Ganesh-RB/AI1103prob-and-randomvariables/Assignment4/codes
\end{lstlisting}
%
and latex-tikz codes from 
%
\begin{lstlisting}
https://github.com/Ganesh-RB/AI1103prob-and-randomvariables/Assignment4
\end{lstlisting}

\section{Problem}
{
\centering CSIR UGC NET EXAM (Dec 2012) Q 51

}
Suppose $X_1,X_2,X_3,X_4$ are i.i.d random variables taking values $1$ and $-1$ with probability $1/2$ each. Then $E\brak{X_1+X_2 +X_3 +X_4}^4$ equals
\begin{multicols}{4}
\begin{enumerate}
    \item $4$
    \item $76$
    \item $16$
    \item $12$
\end{enumerate}
\end{multicols}


\section{Solution}
\begin{theorem}
If ${ X_{1},\dots ,X_{n}}$ are i.i.d. random variables, all Bernoulli trials with success probability p, then their sum is distributed according to a binomial distribution with parameters $n$ and $p$

{\centering
${\displaystyle \sum _{k=1}^{n}X_{k}\sim  {B} (n,p)}$

}
\label{binom_theorem}
\end{theorem}

\begin{corollary}
For a binomial random variable $X$ with parameters $n$ and $p$ and $q=1-p$
\begin{align}
    E\brak{X}=& n p 
    \\E\brak{X^2}=& n p \brak{n p+q}
    \\E\brak{X^3}=& n p \brak{n^{2} p^{2} + 3 n p q - 2 p q + q}
    \\
    \begin{split}
         E\brak{X^4}=&n p \Bigl(n^{3} p^{3} + 6 n^{2} p^{2} q
        - 11 n p^{2} q  \\& + 7 n p q - 6 p q^{2} + q\Bigr)
    \end{split}
\end{align}
\label{corolary-momentof-Binom}
\end{corollary}
Given that $X_i$ are i.i.d random variable for $i \in \cbrak{1,2,3,4} $ 
with,
\begin{align}
    \pr{X_i=+1}=&\frac{1}{2}
    \\\pr{X_i=-1}=&\frac{1}{2}
\end{align}
\begin{lemma}
Let random variables $M_i=\frac{X_i+1}{2}$ then,
$M_i$ are Bernoulli random variables with $p=0.5$
\end{lemma}
\begin{proof}
\begin{align}
    \pr{M_i=1}=&\pr{X_i=1}=\frac{1}{2}
    \\\pr{M_i=0}=&\pr{X_i=-1}=\frac{1}{2}
\end{align}
$\therefore$ $M_i$ are Bernoulli random variables with $p=0.5$
\end{proof}


Let assume random variable $Y$ as
\begin{align}
Y=\sum_{i=1}^{4}X_i
\end{align} 

similarly let 
\begin{align}
Z=&\sum_{i=1}^{4}M_i
\end{align}


\begin{corollary}
$Z$  is Binomial random variable , \\from theorem \ref{binom_theorem}

{\centering
${\displaystyle Z \sim  {B} (4,0.5)}$

}
\label{cor-Z_as_binom}
\end{corollary}

\begin{align}
Z=&\sum_{i=1}^{4} \frac{X_i+1}{2}
\\ Z=&\frac{Y}{2} + 2
\end{align}



\begin{align}
    {Y^4}=16{(Z-2)^4}
\end{align}
\begin{align}
    {Y^4}=16\brak{Z^4-8z^3+24Z^2-32Z+16}
\end{align}

\begin{align}
\begin{split}
    E\brak{Y^4}=&16\Bigl(E\brak{Z^4}-8E\brak{z^3}
    \\&+24E\brak{Z^2}-32E\brak{Z}+16\Bigr)
\end{split}    
\end{align}
From corollary \ref{corolary-momentof-Binom} and  \ref{cor-Z_as_binom}
\begin{align}
    E\brak{Y^4}=&40 &
\end{align}

{\centering
$\mbf{\therefore E\brak{X_1+X_2 +X_3 +X_4}^4=40}$

}

\end{document}
