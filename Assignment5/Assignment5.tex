\documentclass[journal,12pt,twocolumn]{IEEEtran}

\usepackage{setspace}
\usepackage{gensymb}
\singlespacing
\usepackage[cmex10]{amsmath}

\usepackage{amsthm}

\usepackage{mathrsfs}
\usepackage{txfonts}
\usepackage{stfloats}
\usepackage{bm}
\usepackage{cite}
\usepackage{cases}
\usepackage{subfig}

\usepackage{longtable}
\usepackage{multirow}

\usepackage{enumitem}
\usepackage{mathtools}
\usepackage{steinmetz}
\usepackage{tikz}
\usepackage{circuitikz}
\usepackage{verbatim}
\usepackage{tfrupee}
\usepackage[breaklinks=true]{hyperref}
\usepackage{graphicx}
\usepackage{tkz-euclide}

\usetikzlibrary{calc,math}
\usepackage{listings}
    \usepackage{color}                                            %%
    \usepackage{array}                                            %%
    \usepackage{longtable}                                        %%
    \usepackage{calc}                                             %%
    \usepackage{multirow}                                         %%
    \usepackage{hhline}                                           %%
    \usepackage{ifthen}                                           %%
    \usepackage{lscape}     
\usepackage{multicol}
\usepackage{chngcntr}

\DeclareMathOperator*{\Res}{Res}

\renewcommand\thesection{\arabic{section}}
\renewcommand\thesubsection{\thesection.\arabic{subsection}}
\renewcommand\thesubsubsection{\thesubsection.\arabic{subsubsection}}

\renewcommand\thesectiondis{\arabic{section}}
\renewcommand\thesubsectiondis{\thesectiondis.\arabic{subsection}}
\renewcommand\thesubsubsectiondis{\thesubsectiondis.\arabic{subsubsection}}


\hyphenation{op-tical net-works semi-conduc-tor}
\def\inputGnumericTable{}                                 %%

\lstset{
%language=C,
frame=single, 
breaklines=true,
columns=fullflexible
}
\begin{document}


\newtheorem{theorem}{Theorem}[section]
\newtheorem{problem}{Problem}
\newtheorem{proposition}{Proposition}[section]
\newtheorem{lemma}{Lemma}[section]
\newtheorem{corollary}[theorem]{Corollary}
\newtheorem{example}{Example}[section]
\newtheorem{definition}[problem]{Definition}

\newcommand{\BEQA}{\begin{eqnarray}}
\newcommand{\EEQA}{\end{eqnarray}}
\newcommand{\define}{\stackrel{\triangle}{=}}
\bibliographystyle{IEEEtran}
\raggedbottom
\setlength{\parindent}{0pt}
\providecommand{\mbf}{\mathbf}
\providecommand{\pr}[1]{\ensuremath{\Pr\left(#1\right)}}
\providecommand{\qfunc}[1]{\ensuremath{Q\left(#1\right)}}
\providecommand{\sbrak}[1]{\ensuremath{{}\left[#1\right]}}
\providecommand{\lsbrak}[1]{\ensuremath{{}\left[#1\right.}}
\providecommand{\rsbrak}[1]{\ensuremath{{}\left.#1\right]}}
\providecommand{\brak}[1]{\ensuremath{\left(#1\right)}}
\providecommand{\lbrak}[1]{\ensuremath{\left(#1\right.}}
\providecommand{\rbrak}[1]{\ensuremath{\left.#1\right)}}
\providecommand{\cbrak}[1]{\ensuremath{\left\{#1\right\}}}
\providecommand{\lcbrak}[1]{\ensuremath{\left\{#1\right.}}
\providecommand{\rcbrak}[1]{\ensuremath{\left.#1\right\}}}
\theoremstyle{remark}
\newtheorem{rem}{Remark}
\newcommand{\sgn}{\mathop{\mathrm{sgn}}}
%\providecommand{\abs}[1]{\left\vert#1\right\vert}
\providecommand{\res}[1]{\Res\displaylimits_{#1}} 
%\providecommand{\norm}[1]{\left\lVert#1\right\rVert}
%\providecommand{\norm}[1]{\lVert#1\rVert}
\providecommand{\mtx}[1]{\mathbf{#1}}
%\providecommand{\mean}[1]{E\left[ #1 \right]}
\providecommand{\fourier}{\overset{\mathcal{F}}{ \rightleftharpoons}}
%\providecommand{\hilbert}{\overset{\mathcal{H}}{ \rightleftharpoons}}
\providecommand{\system}{\overset{\mathcal{H}}{ \longleftrightarrow}}
	%\newcommand{\solution}[2]{\textbf{Solution:}{#1}}
\newcommand{\solution}{\noindent \textbf{Solution: }}
\newcommand{\cosec}{\,\text{cosec}\,}
\providecommand{\dec}[2]{\ensuremath{\overset{#1}{\underset{#2}{\gtrless}}}}
\newcommand{\myvec}[1]{\ensuremath{\begin{pmatrix}#1\end{pmatrix}}}
\newcommand{\mydet}[1]{\ensuremath{\begin{vmatrix}#1\end{vmatrix}}}
\numberwithin{equation}{subsection}
\makeatletter
\@addtoreset{figure}{problem}
\makeatother
\let\StandardTheFigure\thefigure
\let\vec\mathbf
\renewcommand{\thefigure}{\theproblem}
\def\putbox#1#2#3{\makebox[0in][l]{\makebox[#1][l]{}\raisebox{\baselineskip}[0in][0in]{\raisebox{#2}[0in][0in]{#3}}}}
     \def\rightbox#1{\makebox[0in][r]{#1}}
     \def\centbox#1{\makebox[0in]{#1}}
     \def\topbox#1{\raisebox{-\baselineskip}[0in][0in]{#1}}
     \def\midbox#1{\raisebox{-0.5\baselineskip}[0in][0in]{#1}}
\vspace{3cm}



\title{ASSIGNMENT 5}
\author{Ganesh Bombatkar - CS20BTECH11016}
\maketitle
\newpage
\bigskip
\renewcommand{\thefigure}{\theenumi}
\renewcommand{\thetable}{\theenumi}
Download all python codes from 
\begin{lstlisting}
https://github.com/Ganesh-RB/AI1103prob-and-randomvariables/Assignment5/codes
\end{lstlisting}
%
and latex-tikz codes from 
%
\begin{lstlisting}
https://github.com/Ganesh-RB/AI1103prob-and-randomvariables/Assignment5
\end{lstlisting}
\section{Problem}
For $n \geq 1$, let $X_n$ be a Poisson random  variable with mean $n^2$.
 Which of the following's are equal to
 $\displaystyle{\frac{1}{\sqrt{2\pi}} \int \limits_2^{\infty} e^{-x^2/2}\,dx}$
\begin{enumerate}
    \item $\lim \limits_{n \to \infty} $ \pr{X_n > (n+1)^2}
    \item $\lim \limits_{n \to \infty} $ \pr{X_n \leqslant (n+1)^2}
    \item $\lim \limits_{n \to \infty} $ \pr{X_n<(n-1)^2}
    \item $\lim \limits_{n \to \infty} $ \pr{X_n<(n-2)^2}
\end{enumerate}


\section{Solution}
Let $Y_i$ be a Poisson random variable with mean $1$ for $i$ $\in$ $\brak{1,n^2}$
\vspace{0.5cm}
\\By additive property of Poisson distribution
\begin{align}
   { \sum \limits_{i}^{n^2} Y_i = X_n   }  
\end{align}

By central limit theorem 

\begin{align}
    \lim  \limits_{n \to \infty} \frac{Y_1+Y_2+..+Y_{n^2} - n^2}{n} = \mathcal{N}(0,1)
    \\ \implies \lim  \limits_{n\to \infty} \frac{X_n-n^2}{n} = \mathcal{N}\brak{0,1}
\end{align}
Here, $\mathcal{N}\brak{0,1}$ is normal distribution with unit mean and variance

Now 
{
\begin{align}
    \pr{X_n>k} =& \pr{\frac{X_n-n^2}{n}>\frac{k-n^2}{n}}
    \\=& \pr{\mathcal{N}\brak{0,1}>\frac{k-n^2}{n}}
    \\ =& Q\brak{\frac{k-n^2}{n}}
    \label{eq-cdf}
\end{align}
}
\vspace{0.2cm}
\\Here
\begin{align}
    Q\brak{X}=1-Q\brak{-x}
    \label{eq-Qx}
\end{align}
\begin{center}
    $\because \,{\frac{1}{\sqrt{2\pi}} \int \limits_{-x}^{\infty} e^{-x^2/2}\,dx}\,=\,{\frac{1}{\sqrt{2\pi}} \int \limits_{-\infty}^x e^{-x^2/2}\,dx}$
    \\ and \quad $\frac{1}{\sqrt{2\pi}} \int \limits_{-\infty}^{\infty} e^{-x^2/2}\,dx\,=1$
\end{center}
Also
\begin{align}
    \frac{1}{\sqrt{2\pi}} \int \limits_2^{\infty} e^{-x^2/2}\,dx=& Q\brak{2}
\end{align}
\vspace{0.4cm}
\\
{\small
\begin{enumerate}
\item $\lim \limits_{n \to \infty} $ \pr{X_n > (n+1)^2}
\begin{align}
    \lim \limits_{n \to \infty} \pr{X_n > (n+1)^2} =& \lim \limits_{n \to \infty}Q \brak{\frac{(n+1)^2-n^2}{n}}
    \\=&Q\brak{2}
\end{align}
$\mathbf{\therefore}$ \textbf{Option 1 is correct}
\vspace{0.4cm}
\item $\lim \limits_{n \to \infty} $ \pr{X_n \leqslant (n+1)^2}
\begin{align}
    \lim \limits_{n \to \infty}  \pr{X_n \leqslant (n+1)^2} =&1-\lim \limits_{n \to \infty}  \pr{X_n > (n+1)^2}
    \\=&1- \lim \limits_{n \to \infty}Q \brak{\frac{(n+1)^2-n^2}{n}}
    \\=&1- Q \brak{2}
    \\ =& Q\brak{-2} > Q\brak{2}
    % \\ \lim \limits_{n \to \infty}  \pr{X_n \leqslant (n+1)^2} >&  
\end{align}

$\mathbf{\therefore}$ \textbf{Option 2 is incorrect}
\vspace{0.4cm}
\item $\lim \limits_{n \to \infty} $ \pr{X_n < (n-1)^2}
\begin{align}
    \lim \limits_{n \to \infty}  \pr{X_n < (n-1)^2} =& 1-\lim \limits_{n \to \infty}  \pr{X_n \geq (n-1)^2}
    \\=&1-\lim \limits_{n \to \infty}Q \brak{\frac{(n-1)^2-n^2}{n}}
    \\=& 1-Q \brak{-2}
    \\=& Q\brak{2}
\end{align}
$\mathbf{\therefore}$ \textbf{Option 3 is also correct}
\vspace{0.4cm}
\item $\lim \limits_{n \to \infty} $ \pr{X_n < (n-2)^2}
\begin{align}
    \lim \limits_{n \to \infty}  \pr{X_n < (n-2)^2} =& 1-\lim \limits_{n \to \infty}  \pr{X_n \geq (n-2)^2}
    \\=&1-\lim \limits_{n \to \infty}Q \brak{\frac{(n-2)^2-n^2}{n}}
    \\=&1- Q \brak{-4}
    \\=& Q\brak{4} < Q \brak{2}
\end{align}
$\mathbf{\therefore}$ \textbf{Option 4 is incorrect}
\end{enumerate}
}%


\end{document}
