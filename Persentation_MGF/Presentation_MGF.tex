\documentclass{beamer}
\usepackage{listings}
\lstset{
%language=C,
frame=single, 
breaklines=true,
columns=fullflexible
}

% \graphicspath{./figures/}

\usepackage{subcaption}
\usepackage{url}
\usepackage{tikz}
\usepackage{tkz-euclide} % loads  TikZ and tkz-base
%\usetkzobj{all}
\usetikzlibrary{calc,math}
\usepackage{float}
\newcommand\norm[1]{\left\lVert#1\right\rVert}
\renewcommand{\vec}[1]{\mathbf{#1}}
\usepackage[export]{adjustbox}
\usepackage[utf8]{inputenc}
\usepackage{amsmath}



\providecommand{\mbf}{\mathbf}
\providecommand{\pr}[1]{\ensuremath{\Pr\left(#1\right)}}
\providecommand{\qfunc}[1]{\ensuremath{Q\left(#1\right)}}
\providecommand{\sbrak}[1]{\ensuremath{{}\left[#1\right]}}
\providecommand{\lsbrak}[1]{\ensuremath{{}\left[#1\right.}}
\providecommand{\rsbrak}[1]{\ensuremath{{}\left.#1\right]}}
\providecommand{\brak}[1]{\ensuremath{\left(#1\right)}}
\providecommand{\lbrak}[1]{\ensuremath{\left(#1\right.}}
\providecommand{\rbrak}[1]{\ensuremath{\left.#1\right)}}
\providecommand{\cbrak}[1]{\ensuremath{\left\{#1\right\}}}
\providecommand{\lcbrak}[1]{\ensuremath{\left\{#1\right.}}
\providecommand{\rcbrak}[1]{\ensuremath{\left.#1\right\}}}
\usetheme{Boadilla}

\title{Moment Generating Function}
\author{Ganesh Bombatkar-CS20BTECH11016}

\institute{}
\date{\today}
\begin{document}


\begin{frame}
\titlepage
\end{frame}
\section{Problem}
\begin{frame}{}
\begin{block}{Problem}    
Suppose $X_1,X_2,X_3,X_4$ are i.i.d random variables taking values $1$ and $-1$ with probability $1/2$ each. Then $E\brak{X_1+X_2 +X_3 +X_4}^4$ equals
\begin{enumerate}
    \item $40$
    \item $76$
    \item $16$
    \item $12$
\end{enumerate}
\end{block}
\end{frame}

\section{Solution}
\begin{frame}{Solution}
\begin{block}{}
$X_i$, for $i \in \cbrak{1..4}$ are i.i.d random variables with
\begin{align}
    \pr{X_i=+1}=&\frac{1}{2}
    \\\pr{X_i=-1}=&\frac{1}{2}
\end{align}
\end{block}

\begin{block}{}
Let $Y=X_1+X_2 +X_3 +X_4$
\begin{table}[h]
    \centering
    \begin{tabular}{||c||c|c|c|c|c|c||}
         \hline
         \hline
          $Y=k$ &  $-4$ & $-2$ & $0$ & $2$ & $4$ \\
         \hline
          $p_Y(k)$ & ${1}/{16}$ & ${1}/{4}$ & ${3}/{8}$ & ${1}/{4}$ & ${1}/{16}$\\
         \hline
         \hline
    \end{tabular}
    \caption{pmf Y}
    \label{tab:probability_Y}
\end{table}
\end{block}
\end{frame}

\begin{frame}{}
    \begin{block}{Expectation}
    Expectation of discrete random variable $X$ denoted with $E(X)$
    \begin{align}
        E(x)=&\sum_{k}k\times p_X(k)
        \\E\brak{X^n}=&\sum_{k}k^n\times p_X(k)
    \end{align}
    
    \end{block}
    \begin{block}{}
        \begin{align}
            E\brak{Y^4}=&\sum_{k}k^4\times p_X(k)
            \\ =& 256\times \frac{1}{16}+              16\times \frac{1}{4}+             0\times \frac{3}{8}+             16\times \frac{1}{4}+               256\times \frac{1}{16}
            \\E\brak{Y^4}=&40
        \end{align}
    \end{block}
\end{frame}


\begin{frame}{Moment Generating Function}
    \begin{definition}[Moment]
    $n^{\text{th}}$ Moment of a random variable X is $E\brak{X^n}$ 
    \end{definition}
    
    \begin{block}{}
    Moments are useful in describing shape of any distribution
    \begin{itemize}
        \item First moment represent mean.
        \item Second represents variance.
        \item Third represent skewness, indicates any asymmetric ‘leaning’ to either left or right.
    \end{itemize}
    \end{block}
\end{frame}

\section{Moment Generating function}
\begin{frame}{Moment Generating Function}
    \begin{definition}[Moment generating function]
    Moment Generating Function (MGF) of a random variable $X$ is a function $M_X(t)$ defined as
    \begin{align}
        M_{X}\brak{t} = E\brak{e^{t X}}
    \end{align}
    $M_X(t)$ exist, if there exists $a$ positive constant $a$ such that $M_X(t)$ is finite for all $t \in [-a,a]$
\end{definition}

\begin{block}{}
If $X$ is a discrete random variable 
    \begin{align}
        M_{X}\brak{t}  = \sum_{k=-\infty}^{\infty} e^{t k}\times p_X\brak{k}
    \end{align}
\end{block}
\end{frame}

\begin{frame}{}
    \begin{block}{Taylor series}
    Taylor series for $e^x$
\begin{align}
e^x=1+x+\frac{x^2}{2!}+\frac{x^3}{3!}+...=\sum_{k=0}^{\infty} \frac{x^k}{k!}.
\end{align}

\end{block}
\vspace{4pt}
\begin{theorem}
$n^{\text{th}}$ Moment of $X$ \brak{E\brak{X^n}} is coefficient of $\cfrac{t^n}{n!}$ in Taylor expansion of $M_X(t)$
\label{nth-Moment-Theorem}
\end{theorem}

\end{frame}

\begin{frame}{}
    \begin{proof}
Taylor series of $e^{t X}$ 
\begin{align}
    e^{t X}=&\sum_{k=0}^{\infty} \frac{X^k{t}^k}{k!}
\end{align}
Taking expectation on both side
\begin{align}
    E\brak{e^{t X}}=&\sum_{k=0}^{\infty}E\brak{X^k} \frac{{t}^k}{k!}
\end{align}
$\therefore$ $E\brak{X^n}$ is coefficient of $\frac{{t}^n}{n!}$ in Taylor expansion of $M_X(t) \equiv E\brak{e^{t X}}$  
\end{proof}
\end{frame}

\subsection{solution using MGF}
\begin{frame}{Solution using MGF}
\begin{block}{}
$X_i$ are i.i.d random variable for $i \in \cbrak{1,2,3,4} $ 
with,
\begin{align}
    \pr{X_i=+1}=&\frac{1}{2}
    \\\pr{X_i=-1}=&\frac{1}{2}
\end{align}
\end{block}
\begin{block}{}
By defining Moment generating function of $X_i$
\begin{align}
    M_{X_i}\brak{t}=\frac{1}{2}\times e^{-t} + \frac{1}{2} \times e^{t}=\frac{e^{t}+e^{-t}}{2}
\end{align}
\end{block}
\end{frame}

\begin{frame}{}
\begin{block}{}
Let $Y=\sum_{i=0}^{4}X_i$ , then by convolution
\begin{align}
    M_{Y}\brak{t}=& \prod_{i=1}^{4} M_{X_i}\brak{t} 
    \\M_{Y}\brak{t} =& \brak{\frac{e^{t}+e^{-t}}{2}}^4
    \\M_{Y}\brak{t} =& \frac{e^{4t}+4e^{2t}+6+4e^{-2t}+e^{-4t}}{16}
\end{align}
\end{block}
\begin{block}{}
By using Taylor expansion of $e^{4t}$, $e^{2t}$, $e^{-2t}$ and $e^{-4t}$
\begin{align}
    M_Y(t)=1+ \frac{1}{16}\brak{ \sum_{k=1}^{\infty} \brak{4^k+4(2)^k+4(-2)^k+(-4)^k}\frac{t^k}{k!} }
    \label{eqn:M_Y(t)}
\end{align}
\end{block}
\end{frame}

\begin{frame}{}
By Theorem \ref{nth-Moment-Theorem} and eqn: \eqref{eqn:M_Y(t)}
\begin{block}{}
\begin{align}
    E\brak{Y^n}=&\frac{4^n+4(2)^n+4(-2)^n+(-4)^n}{16} 
\end{align}

\begin{align}
    E\brak{Y^4}=&\frac{4^4+4(2)^4+4(-2)^4+(-4)^4}{16} 
    \\E\brak{Y^4}=&\frac{640}{16}=40
\end{align}

{\centering
$\mbf{\therefore E\brak{X_1+X_2 +X_3 +X_4}^4=40}$

}
\end{block}
\end{frame}
\end{document}
